\documentclass[fleqn]{article}
\usepackage{graphicx}
\usepackage{curves}
\usepackage{epic}
\usepackage{eepic}
\usepackage{amsmath}
\usepackage{amssymb}
%\usepackage{eepicemu}% bad!!
\usepackage[dvipdfmx,usenames]{color}
\usepackage{array}
\usepackage{colortbl}
\usepackage[usenames]{color}
\renewcommand{\arraystretch}{2.4}

\begin{document}
\title{星座 クイズ }
\maketitle
\hoffset = 0pt
\voffset = 0pt
\topmargin=0pt
\textheight=21cm
\headheight=0pt
\headsep=0pt
\newcommand{\CHo}{\makebox(0,0){$\bullet$}}
\newcommand{\CHc}{\makebox(0,0){$\circ$}}
\newcommand{\CHb}{\makebox(0,0){$\bigcirc$}}


\renewcommand{\descriptionlabel}[1]
{\hspace{\labelsep}\textsf{#1}}
\begin{description}
\item[Q 1]{大熊座の北斗七星のうち2等星はいくつあるか?}

\item[Q 2]{七夕伝説の牽牛星と織女星はこと座のヴェガと わし座
のアルタイルが対応します。牽牛星はどちらか?
}
\item[Q 3]{88星座のうち最も広い星座は?}

\item[Q 4]{領域がつながっていない星座はどれか?
またそれを分断した星座は?
}

\item[Q 5]{北斗七星の柄のカーブをそのまま南に延ばして得られる
「春の大曲線」の主役の星2つを示せ。}

\item[Q 6]{秋の四辺形の3隅はペガサス座です。左上隅は何座の星か?}


\item[Q 7]{21ある1等星のtop three を示しなさい。}

\item[Q 8]{
オリオン座には四辺形とその中の
三ツ星があります。この7つの星のうち
で1等星を示しなさい}

\item[Q 9]{冬のダイヤモンドと呼ばれる6ケの星を
図で示しなさい。いずれも1等星です。

}
\item[Q10]{野尻抱影が「三ッ矢サイダ-」の商標から命名した
三ッ矢は何座にあるか?

}

\item[Q11]{紀元前3000年頃造られた、エジプトのギゼーのケオフス王の
大ピラミッドには、傾斜角27°、長さ116mのトンネルが北に向かって
いる。当時ある星を観測していたと思われる。その星は?


}


\item[Q12]{北斗七星の柄から2番目の星ミザールの傍に小さい星が
あり、アラビアでは兵士の眼の検査\footnote{
星座ガイドブック春夏編p49} にこの星の名は?
}

\item[Q13]{WikiPedeaでの「星座一覧」はラテン語名、略号 そして 日本語で
ソートできる。日本語でソートしたら一番始めにくる星座は?
}
\item[Q14]{元旦のTVで1月4日未明に 「しぶんぎ座で多くの流星が
みれる」と報じました。現在しぶんぎ座は存在しない。どこにあるのか?
}
\item[Q15]{動物の名前に大小をつけて対の星座があります。その
動物たちは?
}

\item[Q16]{カシオペア座から北極星を探す方法はいくつかありますが、Wの5点
すべてを用いて作図しなさい。
}
\item[Q17]{冥王星は2020年現在やぎ座にいますが、過去100年間にいたことがある
星座を示しなさい。
}

\item[Q18]{現在の88星座は1922年に決められたが、それ以前にトレミーの
48星座があった。トレミーとはいつの時代の人か?
}
\item[Q19]{太陽が通る道を黄道といいます。この道にある黄道十二星座
を列挙しなさい。
}

\item[Q20]{へびつかい座は黄道十二星座に含まれていませんか、黄道が
通ります。前問のどこに挿入されますか?
}
\item[Q21]{次の表は一等星の一覧です。日本で見ることができない星を
マークしなさい。\\



{\renewcommand\arraystretch{1.0}
\begin{tabular}{|r|l|c|l|l|}
\hline
No&視等級&星座&バイエル符号&名称\\
\hline
1&-1.46&おおいぬ座&α星&Sirius\\
\hline
2&-0.7&りゅうこつ座&α星&Canopus\\
\hline
3&-0.1&ケンタウルス座&α星&Rigil Kentaurus\\
\hline
4&-0.05&うしかい座&α星&Arcturus\\
\hline
5&0.03&こと座&α星&Vega\\
\hline
6&0.08&ぎょしゃ座&α星&Capella\\
\hline
7&0.13&オリオン座&β星&Rigel\\
\hline
8&0.37&こいぬ座&α星&Procyon\\
\hline
9&0.42&オリオン座&α星&Betelgeuse\\
\hline
10&0.46&エリダヌス座&α星&Achernar\\
\hline
11&0.60&ケンタウルス座&β星&Hadar(Agena)\\
\hline
12&0.76&わし座&α星&Altair\\
\hline
13&0.81&みなみじゅうじ座&α星&Acrux\\
\hline
14&0.86&おうし座&α星&Aldebaran\\
\hline
15&0.91&さそり座&α星&Antares\\
\hline
16&0.97&おとめ座&α星&Spica\\
\hline
17&1.14&ふたご座&β星&Pollux\\
\hline
18&1.16&みなみのうお座&α星&Fomalhaut\\
\hline
19&1.25&はくちょう座&α星&Deneb\\
\hline
19&1.25&みなみじゅうじ座&β星&Mimosa\\
\hline
21&1.40&しし座&α星&Regulus\\
\hline
\end{tabular}
}}
\item[Q22] {ギリシャ神話の登場人物で星座になった方がいます。\\
エリダヌス、オリオン、アンドロメダ、ペガサス、ケンタルウス、
ヘラクレス、ペルセウス、カシオペア\\
このうち女性は誰ですか?

}
\item[Q23] {将棋の駒の型をした五角形のかなり大きくて、一等星も
含む星座があります。何座?
}
\item[Q24]{やぎ座でおこった海王星に関する美談をご存知ですか?
登場するのは アダムス、エアリー、アダムス、ガレーです。
}

\item[Q 25]{さそり座は1等星アンタレスを扇の要とし、β、δ、χの
三点で扇状を造ります。扇の先端の近くに、小さな三連星があります。
この三連星の呼び名は?

}

\item [Q 25]{ペルセウス座のβアルゴルは68時間を周期で、明るさを
2.3等星から3.5等星に変化します。暗い時間は短い。
アルゴルの変光の仕掛けは?

}
\item[Q 26]{くじら座のβミラも変光星です。明るさは2等星から10等星
まで変化します。6等星以下なら肉眼では見えません。この星には
「不思議なもの」の意味のMiraがつけられました。この星の変化は
星の膨張と縮小によるものです。では変光の周期は?


}
\end{description}
\newpage


\setlength\unitlength{1truecm}






\newpage
{\Large 解答}

\begin{description}
%-----------------------------------------
\item[A 1]{ 北斗七星の2等星は6コある、。 fig 1\\

\begin{minipage}{5cm}
\begin{picture}(10,5)
\Thicklines
\begin{dashjoin}{,2}
\jput(9.9,4.3){\CHo α}
\jput(10.5,2.5){\CHo β}
\jput(7.9,1.0){\CHo γ}
\jput(6.0,2.3){\CHo δ}
\jput(4.5,2.2){\CHo ε}
\jput(2.8,2.3){\CHo ξ}
\jput(0.6,1.0){\CHo η}
\end{dashjoin}
\end{picture}
\end{minipage}

\begin{minipage}{3cm}
α Dobhe(2)\\
β Merach(2)\\
γ Phecda(2)\\
δ Megres(3)\\
ε Alioth(2)\\
ξ Mizar(2)\\
η Alcaido(2)
\end{minipage}
}
%-----------------------------------------
\item[A 2]{アルタイルが牽牛星でヴェガが織女星
}

\item[A 3]{うみへび座が 1302平方度で一番広い。全体\footnote{$ 2\pi r=360 , s=4\pi r^2$}の3\% にあたる
}



\item[A 4]{へび座が頭部と尾部に分断されている。分断」したのは「へびつかい座」。
}
\item[A 5]{
アークツルス(うしかい座)とスピカ(おとめ座)
}
\item[A 6]{アンドロメダ座のα
}

\item[A 7]{シリウス、カノープス、リゲル・ケンタルス

\newpage


\item[A 8]
{
Orionには 1等星は2ツある\\
\begin{minipage}{7cm}

\begin{picture}(10,8.5)
\put(0.7,7.7){\CHo α}
\put(4.5,1.6){\CHo β}
\put(3.5,7.3){\CHo γ}
\put(3.0,4.8){\CHo δ}
\put(2.6,4.3){\CHo ε}
\put(2.1,4.0){\CHo ξ}
\put(1.4,1.0){\CHo κ}
\end{picture}
\end{minipage}
\begin{minipage}{5cm}{
 α Betercuse(1)}\\
 β Rigel(1)\\
 γ Bellatrix(2)\\
 δ Mintaka(2)\\
 ε Alnilam(2)\\
 ξ Alnitak(2)\\
 κ Saiph(2)
  \end{minipage}



}

\item[A 9]{

冬のダイヤモンド\\

\begin{minipage}{3cm}

\begin{picture}(10,10)
\begin{dashjoin}{,2}
\jput(9.6 ,7.4){\CHo 1}
\jput(7.1,9.6){\CHo 2}
\jput(0.4,9.4){\CHo 3}
\jput(0.7,5.0){\CHo 4}
\jput(3.4,0.8){\CHo 5}
\jput(7.7,2.5){\CHo 6}
\jput(9.6,7.4){ }
\end{dashjoin}
\end{picture}
\end{minipage}
\begin{minipage}{4cm}
1 Aldebaran(Boo おうし)\\
2 Capella(Aur ぎょしゃ)\\
3 Pollux(Gem ふたご)\\
4 Procyon(CMi こいぬ)\\
5 Sirius(CMa おおいぬ)\\
6 Rigel(Ori オリオン)
\end{minipage}
}


}
\newpage

\item[A 10]{ みずがめ座 \\
\begin{minipage}{6cm}

\begin{picture}(10,2.5)
\begin{dashjoin}{,2}
\jput(1.4,1.3){}
\jput(2.1,0.8){\CHo γ}
\end{dashjoin}
\begin{dashjoin}{,2}
\jput(0.9 ,1.2){\CHo η}
\jput(1.4,1.3){\CHo ξ}
\jput(1.9,1.9){\CHo π}
\end{dashjoin}
\begin{dashjoin}{,2}
\jput(3.5,1.3){\CHo α}
\jput(4.9,0.6){\CHo o}
\end{dashjoin}



\end{picture}
\end{minipage}




}
\item[A 11]{ りゅう座のα ツバーン。}
\item[A 12]{アルゴル}
\item[A 13]{いっかくじゅう座。アンドロメダ座はカタカナなので「ひらがな」の後。
}
\item[A 14]{
四分儀座は1922年に88星座を決めた時に、選から外され、
しぶんぎ座が設定されていた領域はりゅう座の一部となりました。
}

%-----------------------------------------

\item[A 15]{熊、犬、獅子。
略称の後ろ2文字は major と minor を示すMa 、Miが付きます。
例外はしし。

{\renewcommand\arraystretch{1.2}
\begin{tabular}{lc}
おおくま&UMa\\
こぐま&UMi\\
おおいぬ&CMa\\
こいぬ&CMi\\
しし&Leo\\
こじし&LMi\\
\end{tabular}
}%renew


}

\item[A16]{
Casiopea座の作図\\
αβの延長とεδの延長の交点をxとする。\\
xγを5倍延長擦ればpolarisに達する\\

\begin{minipage}{12cm}
\begin{picture}(10,7)
\begin{dashjoin}{.2}
\jput(1.20,1.36){\CHo α}
\jput(0.48,2.64){\CHo β}
\jput(2.00,3.18){\CHo γ}
\jput(2.08,4.27){\CHo δ}
\jput(3.52,4.96){\CHo ε}
\end{dashjoin}
\begin{dottedjoin}{.2}
\jput(10.88,1.36){\CHc polaris}
\jput(0.08,3.44){\CHc x}
\end{dottedjoin}
\drawline(0.08,3.4)(2.08,4.24)
\drawline(0.08,3.4)(0.48,2.64)
\end{picture}
\end{minipage}

\item[A 17]{
冥王星の居場所\\
{\renewcommand\arraystretch{1.0}
\begin{tabular}{ll}
1968&おとめ座\\」
2007&いて座\\
2020&やぎ座\\
2030&みずがめ座\\
\end{tabular}
}


}
\item[A 18]{BC2 の天文学者プトレマイオスの英語よみ
Ptolemy がトレミーになった。
}
}
\item[A 19] {


黄道十二星座\\
{\renewcommand\arraystretch{1.0}
\begin{tabular}{ll}
おひつじ座&(牡羊座、Aries)\\
おうし座 &(牡牛座、Taurus)\\
ふたご座&(双子座、Gemini)\\
かに座&(蟹座、Cancer)\\
しし座&(獅子座、Leo)\\
おとめ座&(乙女座、Virgo)\\
てんびん座&(天秤座、Libra)\\
さそり座&(蠍座、Scorpius)\\
いて座&(射手座、Sagittarius)\\
やぎ座&(山羊座、Capricornus)\\
みずがめ座&(水瓶座、Aquarius)\\
うお座 &(魚座、Pisces)\\
\end{tabular}
}
}
\item[A 20]{てんびん座と さそり座の 間}

\item[A 21]{日本で見えない1等星\\
{\renewcommand\arraystretch{1.0}
\begin{tabular}{llrc}
名称&星座&赤緯&注\\
カーノープス&りゅうこつ&52°&西日本で見える\\
アクラックス&南十字&63°&\\
ミモザ&南十字&61°&\\
アケルケナ&エリダヌス&58°&鹿児島で見える\\
リゲル・ケンタルス&ケンタルス&61°&\\
ハダル&ケンタルス&60°&\\
\end{tabular}
}


}
\item[A 22]{カシオペアとアンドロメダ}

\item[A 23]{ぎょしゃ座 一等星はαカペラ。左下はおうし座に接する。

\begin{minipage}{3cm}

\begin{picture}(10,7.66)
\begin{dashjoin}{,2}
\jput(4.6,5.1){\CHb  α}
\jput(1.6,7.1){\CHo β}
\jput(0.3,4.5){\CHo θ}
\jput(1.8,0.3){\CHo Tauβ}
\jput(5.0,1.2){\CHo ι}
\jput(4.6,5.1){ }
\end{dashjoin}
\end{picture}
\end{minipage}


}
\item[A 24]{
1781年ウイリアム・ハーシェルが天王星を発見して以来、多くの天文学者が
その運動に説明できないものを感じていた。最初にイギリスのアダムスが
計算である惑星の存在を予想し、1843年10月観測をグリニッジ天文台のエアリーに
依頼した、エアリーは机の引き出しにいれたままだった、

フランスのルベルジェ
も同様な計算をしてベルリンの天文台のガレーに送った。ガレーは即座に
やぎ座δあたりで 1846年9月23日に予想されたところに惑星を発見した。

これを知ったエアリーは観測しアダムスのほうが早かったといったが後の
祭り。しかしルベルジェはアダムスを発見者に加えガレーと3人が発見者になったと
いう美談。

}
\item[A 24]{
この三連星はへびつかい座のφχψと呼ばれます。\\

\begin{minipage}{5cm}
\begin{picture}(10,5.66)
\begin{dottedjoin}{,2}
\jput(3.0,3.3){\CHo β}
\jput(3.4,2.2){\CHo δ}
\jput(3.5,0.9){\CHo π}
\end{dottedjoin}
\begin{dottedjoin}{,2}
\jput(0.4,4.7){\CHo φ}
\jput(0.8,4.0){\CHo χ}
\jput(1.0,3.05){\CHo ψ}
\end{dottedjoin}
\put(0.8,0.8){\CHb α}
\put(0.8,0){\makebox(0,0){Sco}}
\put(0.4,2.3){\makebox(0,0){Oph}}
\put(5.0,1.3){\makebox(0,0){Lib}}

\dashline{0.5}(0.0,1.8)(1.3,1.8)(1.3,3.2)(0.65,3.2)
(0.65,3.7)(1.0,3.7)(1.0,6.0)(3.0,6.0)(3.0,4.0)(4.0,4.0)(4.0,0.0)
\end{picture}
\end{minipage}


}
\vspace{1cm}
\item[A 25]{
食変光星が答えです\\

 
\begin{minipage}{4.5cm}
\begin{picture}(10,3.66)

\put(2.0,2.0){\circle{0.8}}
\put(2.0,2.75){\circle*{0.5}}% nuritubusi ???
\put(2.0,1,3){\circle*{0.5}}
\put(0.1,2.0){\circle*{0.5}}
\put(2.0,2.0){\oval(4.0,1.5)}
%\bigcircle[0]{1.0}
\end{picture}
\begin{center}
主星と伴星
\end{center}
\end{minipage}
\begin{minipage}{5cm}

\setlength{\unitlength} {0.82cm}
\begin{picture}(7, 6)
\put(0.5, -0.2){\small \grid(7,6)(1,1){1}}
\put(0,5.75){2.2}
\put(0,4.75){2.4}
\put(0,3.75){2,6}
\put(0,2.75){2.8}
\put(0,1.75){3.0}
\put(0,0.75){3.2}
\put(0, -0.25){3.4}
\put(0.5,-0.7){0}
\put(1.5,-0.7){10}
\put(2.5,-0.7){20}
\put(3.5,-0.7){30}
\put(4.5,-0.7){40}
\put(5.5,-0.7){50}
\put(6.5,-0.7){60}
\put(7.5,-0.7){70}



\linethickness{0.7mm}
\curve(0.5,5.0,
0.8,0.0,
0.9,0.0,
1.5,5.0,
1.6,5.0,
3.5,5.0,
3.6,4.9,
3.8,4.7,
4.0,4.9,
4.3,5.0,
4.6,5.0,
6.5,5.0,
7.0,0.0,
7.1,0.0,
7.7,5.0)
\end{picture}
\begin{center}
\vspace{1cm}
アルゴルの明るさ変化
\end{center}
\end{minipage}


アルゴルは主星と伴星からなる二重星で、伴星が周期68時間
で主星の周りを周ります。伴星は暗い星で主星を隠す位置に
くると3,5星です。逆に
主星が伴星を隠す位置にくると、暗い伴星といえども、光を
出しているので、第二の極(へこみ)がおこります。
主星と伴星がフルに見えると2.3等星です。


}
\item[A 26]{ Mira の周期は330日。
}


%---------------------------------
\end{description}





\end{document}
---------------------------------------------
\setlength{\unitlength} {0.4pt}
\begin{picture}(100,100)
 \put(95,50){\filltype{shade}\ellipse*{35}{20}}
\put(50,50){\circle*{37}}
\end{picture}

