\documentclass[fleqn]{article}
\usepackage{graphicx}
\usepackage{curves}
\usepackage{epic}
\usepackage{eepic}
\usepackage{amsmath}
\usepackage{amssymb}
%\usepackage{eepicemu}% bad!!
\usepackage[dvipdfmx,usenames]{color}
\usepackage{array}
\usepackage{colortbl}
\usepackage[usenames]{color}
\renewcommand{\arraystretch}{2.4}

\begin{document}
\title{百人一首クイズ }
\maketitle
\hoffset = 0pt
\voffset = 0pt
\topmargin=0pt
\textheight=21cm
\headheight=0pt
\headsep=0pt
\newcommand{\CHo}{\makebox(0,0){$\bullet$}}
\newcommand{\CHc}{\makebox(0,0){$\circ$}}
\newcommand{\CHb}{\makebox(0,0){$\bigcirc$}}


\renewcommand{\descriptionlabel}[1]
{\hspace{\labelsep}\textsf{#1}}
\begin{description}
\item[Q 1]{百人一首の中で親子関係は何組あるか?

}
\item[Q 2]{
99番 後鳥羽上皇 (ひともをし)
100番 順徳院 (ももしきや)の歌は百人秀歌にはない。

百人一首の原本とされる百人秀歌には下記の4人の歌以外は
百人一首に現れる。4人とは:
一条院皇后宮、権中納言国信、源俊頼朝臣、権中納言長方で
厳密にいえば源俊頼は差し替えで百人一首に入っているので、
三人が削除されて、両院が追加された。何故100人なのか?

}

\item[Q 3]{
百人一首の歌人は書物をも書いている人が多い。
 
下記の書物の作者を下段から選べ
 
伊勢物語、
歌経標式、
後撰集、
蜻蛉日記、
和漢朗詠集、
和泉式部日記、
源氏物語、
狭衣物語、
枕草子、
詞花集、
奥義抄 、
袋草紙、
愚管抄、
明月記

作者\\
 清少納言 在原業平 左京大夫顕輔 権中納言定家 右大将道綱母  
在原業平 藤原興風 紫式部   大弐三位 和泉式部 
前大僧正慈円 藤原清輔朝臣 大納言公任 謙徳公

}
\end{description}
\newpage
{\Large 解答}

\begin{description}
%-----------------------------------------
\item[A 1]{ 18組ある。

{\renewcommand\arraystretch{1.3}
\begin{tabular}{|c|c|c|}
\hline
親&関係&子\\
\hline
天智天皇(1番)&父と娘&持統天皇(2番)\\
\hline
僧正遍昭(12番)&父と息子&素性法師(21番)\\
\hline
陽成天皇(13番)&父と息子&元良親王(20番)\\
\hline
壬生忠岑(30番)&父と息子&壬生忠見(41番)\\
\hline
文屋康秀(22番)&父と息子&文屋朝康(37番)\\
\hline
藤原定方(25番)&父と息子&藤原朝忠(44番)\\
\hline
清原元輔(42番)&ちちと娘&清少納言(62番)\\
\hline
藤原伊尹(45番)&父と息子&藤原義孝(50番)\\
\hline
平兼盛(40番)&父と娘&赤染衛門(59番)\\
\hline
紫式部(57番)&母と娘&大弐三位(58番)\\
\hline
和泉式部(56番)&母と娘&小式部内侍(60番)\\
\hline
父藤原公任(55番)&と息子&藤原定頼(64番)\\
\hline
源経信(71番)&父と息子&源俊頼(74番)\\
\hline
源俊頼(74番)&ちちと息子&俊恵法師(85番)\\
\hline
藤原顕輔(79番)&父と息子&藤原清輔(84番)\\
\hline
藤原忠通(76番)&父と息子&大僧正慈円(95番)\\
\hline
藤原俊成(83番)&父と息子&藤原定家(97番)\\
\hline
後鳥羽天皇(99番)&父と息子&順徳天皇(100番\\
\hline
\end{tabular}
}}
%-----------------------------------------
\item[A 2]{
どうも計算があわないと言っていただきたい。そう答えは
百人秀歌は101首だったのである。

百人一首の最後の編纂をしたのは、定家の息子の
為家で、両院の歌を入れるのは父の願いでもあったらしい。

百人秀歌が101首なのは、選者自身をいれずに100人
選んだが、まわりから選者もくわえろといわれて101首
になったとか。
}

\item[A 26]{ Mira の周期は330日。
}


%---------------------------------
\end{description}





\end{document}
---------------------------------------------

