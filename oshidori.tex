\documentclass[fleqn]{article}
\begin{document}
\title{おしどりパズル}
\maketitle
\hoffset = 0pt
\voffset = 0pt
\topmargin=0pt
\textheight=21cm
\headheight=0pt
\headsep=0pt

$n>2$として、
nケの白石とnケの黒石が並んでいる。これを白黒交互に並べ替えたい。
例えば

○○○○●●●● を

●○●○●○●○ にする。

操作は相隣る2つに石を同時に動かす。
これをn回で行う。

$n=3$の場合  この場合は特殊で4ケづれる。\\
○○○●●●\\
XX○●●●○○\\
XX○●●XX○●○\\
XXXX●○●○●○\\


$n=4$の場合\\
○○○○●●●●\\
○XX○●●●●○○\\
○●●○XX●●○○\\
○●●○●○●XX○\\
XX●○●○●○●○\\





n=5の場合\\
○○○○○●●●●●\\
○XX○○●●●●●○○\\
○●●○○●●XX●○○\\
○●●○XX●○●●○○\\
○●●○●○●○●XX○\\
XX●○●○●○●○●○\\


n=6の場合\\
○○○○○○●●●●●●\\
○XX○○○●●●●●●○○\\
○●●○○○●XX●●●○○\\
○●●XX○●○○●●●○○\\
○●●○●○●○XX●●○○\\
○●●○●○●○●○●XX○\\
XX●○●○●○●○●○●○\\

n=7の場合\\
○○○○○○○●●●●●●●\\
○XX○○○○●●●●●●●○○\\
○●●○○○○●●●XX●●○○\\
○●●○XX○●●●○○●●○○\\
○●●○●○○●●XX○●●○○\\
○●●○●○●○●○●○●XX○\\
XX●○●○●○●○●○●○●○\\

nの場合ができるのなら n+4 が出来ること
を帰納法で証明する。\\
○○○○$<n+n>$●●●●\\
○XX○$<n+n>$●●●●○○\\
○●●○$<n+n>$XX●●○○\\

nの場合をここで行うと\\
○●●○XX$<n pair>$●●○○\\
○●●○●○$<n pair>$●XX○\\
XX●○●○$<n pair>$●○●○\\



\end{document}
