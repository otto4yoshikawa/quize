\documentclass[fleqn]{article}
\begin{document}
\title{クイズ ギリシャの歴史}
\maketitle
\hoffset = 0pt
\voffset = 0pt
\topmargin=0pt
\textheight=21cm
\headheight=0pt
\headsep=0pt


\begin{enumerate}
\item{19世紀に古代ギリシャ神話のトロヤ遺跡(トロイの木馬)の発掘をした
考古学者は?

\vspace{1zw}A 	シュリーマン
}


\item{ミケーネの遺跡で有名なものは?

\vspace{1zw}A  獅子門

}
\item{ギリシャの島でエーゲ海にある2つの大きな島の名は?

\vspace{1zw}A クレタ島 と エビア島

}
\item{ミケーネで発掘されら粘土板文書の解読を1953年にした人は?

\vspace{1zw}A マイケル・ウエントリス
}
\item{アテネ、ミケーネ、オリンポスの位置関係を述べよ

\vspace{1zw}A アテネはバルカン半島(ギリシャ半島)の南にあり、
ミケーネ、オリンポスはアテネの南にある大きなペロポネソス半島
にある。ミケーネは東、オリンポスは西。
この半島は本土とコリントス地峡でつながっている。
イオニア海がこの半島と沿って深く侵入した海をコリンティアス湾
(コリント湾)と	いった。

}
\item{叙事詩 『イーリアス』と『オデュッセイア』の作者は?
またいつ頃文字になったか?

\vspace{1zw}A ホーマー(ホメロス)

オデュッセイアはイーリアスの続編作品にあたり別人だという説もある。
紀元前6世紀後半のアテナイ(アテネ)において文字化され、紀元前2世紀に
アレキサンドリアにおいて、ほぼ今日の形にまとめられたとされる。
アルファベットはフェニキヤ人の発明であるが、ギリシャで改良された。
フェニキヤ文字の採用と改良は 紀元前850-750と言われる。
叙事詩は伝承されたが、ある程度文字化されている。

}
\item{叙事詩『神統記』の作者は?
1939年からギリシャで発行されていた旧50ドラクマ紙幣に肖像が使用されていた。



\vspace{1zw}A ヘーシオドス 古代ギリシャの叙事詩人。
紀元前	700年頃に活動したと推定される。
『仕事と日々』の作者として知られる。


}
\item{都市オリンピアとオリンポス山は約何キロ離れているか?

\vspace{1zw}A 300 km


}
\item{ソクラテスはアテネの牢獄で毒杯を飲んで死んだ。罪名は?

\vspace{1zw}A\footnote{岩波「哲学思想事典」p983}
 「国家の認める神々を認めずに、他の新奇な神霊をもちこむこと」
と「青年たちを堕落させること」の二つの罪状で告白され、裁判で死刑になった。

ソクラテス自身は著述を行っていないので、その思想は弟子の哲学者プラトンやクセノポン、アリストテレスなどの著作を通じ知られる。プラトン著「ソクラテスの弁明」
山本光男訳がある。


妻のクサンティペは悪妻で有名で、「良妻を持てば幸せ、悪妻を持てば哲学者になれ」
というジョークがある。
}
\item{アテネが戦ったペロポネソス戦争の相手とその結果は?

\vspace{1zw}A\footnote{yahoo知恵袋 ablue26さんの回答}

古代ギリシャのアテナイを中心とした海上帝国とスパルタを中心としたペロポネソス同盟との争いで、見方を変えると民主政対貴族政の戦い、古代ギリシャの主導権争いです。
ペルシア戦争に勝利したギリシャでは対ペルシア帝国の為のデロス同盟がアテナイの為に利用されるようになり、それに反発したスパルタがデロス同盟から離反したポリスなどを支援した事から両者の対立が深まっていきます。
 直接的な原因はコリントスの植民都市ケルキュラが母市に反乱を起こした際、スパルタがコリントスに、アテナイがケルキュラに味方した所から始まります。

アテナイはコリントスには勝てましたが、スパルタ王アルキダモスにアテナイを包囲されているうちに都市内で伝染病が流行って指導者のペリクレスも死亡したことなどから、一時的なニアキスの和を結びます。
その後アテナイの主戦派であったアルキビアデスがシチリア遠征を強行して戦争を再開させた上に大敗を喫し、徐々にデロス同盟から離反が相次ぎ、得意の海戦で粘りを見せたアテナイもアイゴスポタモイの海戦で敗れ、降伏します。

この戦争後ギリシャは慢性的な戦争状態になって衰え、やがてマケドニア王国のフィリッポス2世が覇権を握っていきます。
}
\end{enumerate}
\end{document}