\documentclass[fleqn]{article}
\begin{document}
\title{クイズ ローマの歴史}
\maketitle
\hoffset = 0pt
\voffset = 0pt
\topmargin=0pt
\textheight=21cm
\headheight=0pt
\headsep=0pt

最近 世界の歴史(全16巻)を戴いたので読み出しました
その第2巻 ギリシャとローマ からの抜粋です。

\begin{enumerate}
\item{
イタリア半島の東西両側の海の名を述べよ。\\
\vspace{1zw}A 東がアドリア海 、西がティルレニア海
}
\item{
古代ローマの最初の王は 誰か?\\
\vspace{1zw}A  ロムルス (紀元前600年頃)
}
\item{古代ローマの王は七代続いたが、その選びかたは世襲
  だったか?\\
\vspace{1zw}A No 貴族の合議
}
\item{ローマの北にタルライニアという田舎町がある。タルライニアを
	姓にする古代ローマの王が二人いる。これから彼らの民族を推なさい。
	タルライニアには多くの墳墓が散在しているので。

\vspace{1zw}A 墳墓はエトルリア人の墓だから、王はエトルリア系の人だ。

}

\item{
 カルタゴとローマではどちらが 経度で西にあるか?\\
\vspace{1zw}A カルタゴ
}

\item{古代ローマの七代の王タルクイニウスが追い出されて、ローマは
  共和政になった。西暦何年ごろか?\\
\vspace{1zw}A  6 紀元前50年
}
\item{ 共和政初期に対立するパトリキとプレプスを日本語にすると?\\
\vspace{1zw}A パトリキは名門でプレプスは大衆。
}
\item{
ローマがケルト人に攻められて負けた日を国恥の「禍の日」と
   している。何月何日か?\\
\vspace{1zw}A 7月18日\\
クルト人は掠奪でけで引き揚げた。そうしなかたらローマは
  どうなってたかわからない。
}
\item{ シーザーは川を渡る前に「骰子は投げられた」と言った川の名は?\\
\vspace{1zw}A ルビコン川
}

\item{
カルタゴの名将ハンニバル、マケドニアのアレキサンダー大王、
   クレオパトラ」にほれ込んだアントニウス、この3人の活躍を時代順
   に並べよ。\\
\vspace{1zw}A 

 アレキサンダー大王  前323没、、 
   アレキサンドリア陥落 前314、(自殺)
   ハンニバル      前184没、
}
\item{アントニウスが死んで共和制が終わる。初代のローマ帝政はだれか?\\
\vspace{1zw}A  アウグックス
}
\item{
 ローマの最盛期は五賢帝の時だと言われている。この5人を列挙せよ\\   トラヤキス
   ハドリアヌス
	 アントニヌス・ピウス
   マクレクス・アウレリウス
}
\item{
西暦395年に東西ローマに分裂する。その後どちらが長く続いたか?\\
\vspace{1zw}A  西ローマ帝国は80年 東ローマ帝国1000年
}

\item{
東西ローマの分裂の前に四分割時代があった。二人の正帝と二人の副帝
   を設けた、正帝はアウグストラスといい、副帝はコンスルと言った。
   これを考案して実行したのは誰か?\\
\vspace{1zw}A ディオクレアヌス
}
\item{
キリストの弟子パウロが「quo davis」と言って、ローマにもどった
   時のローマの王は?\\
\vspace{1zw}A6 暴君ネロ
}
\item{
クレオパトラの虜となったアントニウスの妻はクレオパトラより美人だった
   という説がある。彼女の名は?\\
\vspace{1zw}A オクタヴィア
}
\item{
暴君ネロのキリスト教弾圧は有名だが、それよりすごい「最後の大迫害」
  を行った東方正帝は誰か?\\
\vspace{1zw}A  ディオクレディアヌス
}
\item{
8世紀に西ローマ帝国の復興を行ったカール大帝の出身地は?\\
\vspace{1zw}A  フランク
}
\item{
コンスタンディヌス帝は4分割されたローマを再統一する、
   (その後50年で東西ローマの分裂がおこる)。
   彼がキリスト教を国教とする勅令の名は?\\
\vspace{1zw}A  ミラノ勅令
}
\item{
 神聖ローマ帝国はいつ、だれが、どこで創ったか?\\
\vspace{1zw}A  日本では通俗的に、962年ドイツ王オットー1世が
   ローマ教皇ヨハネス12世により、カロリング朝的ローマ帝国の継承者として
   皇帝に戴冠した。
	古代ローマ帝国の後継を称し、その名称は時代とともに幾度も変化した。
	初期 - 西方帝国(カール大帝の帝国は「西ローマ帝国」の復興であった)
	11世紀まで - 帝国(あるいはローマ帝国)
  13世紀 - 神聖ローマ帝国
  1512年 - ドイツ国民の神聖ローマ帝国
}
\item{
313年にコンスタンチィヌス1世が首都をローマからコンスタンティノーブル
   に移した。コンスタンティノーブルは現在はトルコのイスタンブールの前身
   である。そこはプルポロス海峡のヨーロッパ側にあり、その海峡は黒海と
	 マルマラ海を繋ぐ。ではマルマラ海は何海峡によってエーゲ海に通じるか?
\vspace{1zw}A 
2 ダーダスル海峡
}
\item{
西ローマ帝国滅亡後イタリアを支配した王国は?\\
\vspace{1zw}A  
東ゴート王国(ひがしゴートおうこく、Ostrogoths、またはEastern Goths、
	497年 - 553年)は、大王テオドリックによって建国された東ゴート族の王国。
  首都はラヴェンナ。東ローマ帝国の皇帝ゼノンとの同盟により、西ローマ帝国滅亡後 、イタリアのほぼ全域を支配下においた。
}

\item{
民族大移動によって、ローマ領のイべりや半島の侵入した西ゴート族は
   西ローマと戦った。戦いの結末はいかに?\\
\vspace{1zw}A 
  西ゴート王国ができて、ローマの属州になった。
	ローマを一時的に包囲するが、西ローマ皇帝の説得に応じガリアへと撤退した。
	このとき、すでにローマ帝国の支配権が及ばなくなっていたガリアとヒスパニアの
	防衛を条件に、領有を認められたとされている。
	415年にワリア王は南ガリアのトロサ(トゥールーズ)を首都と定め西ゴート王国が
	建国された。
	また、イベリア半島を征服していたヴァンダル族、スエビ族らを討ち、
	褒賞として418年にホノリウス帝から正式に属州アクイタニア(アキテーヌ)を
	与えられた。
	西ゴート王国は東ゴート王国とは争わなかった。300年ほど存続した。
}
\item{
ゲルマン民族大移動に含まれるヴァンダル族はイべりや半島経由で北アフリカ
  に住みつき、カルタゴを支配した。ヴァンダル王国は何年間存続したか?\\
  
\vspace{1zw}A  約100年間   東ローマ帝国によって滅ぼされる。

	かねてよりローマ帝国の復興を企図していた東ローマ帝国の
	皇帝ユスティニアヌス1世は、西ローマ帝国の血を引くヒルデリック王が倒された
	ことを口実にヴァンダル王国に対する戦争を開始し、
	サーサーン朝ペルシャとの戦いで活躍したベリサリウス将軍を派遣した。
	ヴァンダル王国の艦隊のほとんどがサルデニア島の反乱の鎮圧に赴いていることを
	知ったベルサリウス将軍は、 
	迅速に移動してチュニジアに上陸し、カルタゴに入城した。
	533年晩夏、ゲリメル王はカルタゴの南10マイルの所でベリサリウス将軍と戦った。
	(アド・デキムムの戦い)ヴァンダル王国軍は敵を包囲しようとしたが、
	各隊の連携が取れずに失敗し、敗れた。ベルサリウスは、残党と戦う一方で、
	すばやくカルタゴを占領した。
	533年12月15日、カルタゴから20マイルほどのトリカマルムで再び両軍は
	会戦した(トリカマルムの戦い)。そしてまたもやヴァンダル軍は敗れ、
	戦闘の最中にゲリメルの兄弟ツァツォが捕らえられてしまった。
	 ベルサリウスはすぐさま、ヴァンダル王国第二の都市ヒッポに軍を進めた。
	534年、ゲリメルは降伏し、ヴァンダル王国は滅亡した。

}
\item{
共和政ローマ期に反乱を起こしたスパルタクスの職業は何だったか?\\
\vspace{1zw}A  剣闘士
	スパルタのスパルタ教育とは関係ない。
  スパルタクス(ラテン語: Spartacus、生年不詳[1] - 紀元前71年)は、
  共和政ローマ期の剣闘士で、「スパルタクスの反乱」と称される第三次奴隷戦争の
	指導者。

}



\item{
 シーザーが仕組んだ(第1回)三頭政治の三頭は誰々か?\\
 \vspace{1zw}A  シーザー、ポンペイウス、クラックス
}

\item{
シーザーの娘は誰に嫁いだか?\\
\vspace{1zw}A 
 ポンペイウス
}

\item{
「来たり、見たり、勝てり」の科白はいつシーザーがいったか?\\
\vspace{1zw}A 
 アルメニアのミトラダテスの子ファルナケスをゼラの戦いで破った時
   カエサルは元老院に対して「Veni, Vidi, Vici (来た、見た、勝った)」
      という有名な報告をした
}


\item{
 シーザーが刺されたとき、最後の言葉は?\\
 \vspace{1zw}A  「ブルーツスおまえもか」
   暗殺の首謀者ブルーツスはシーザの愛人との間にできた子という
   説もある。
}


\item{
 シーザーには3人の女性がいた。一人は妻カルプニア、二人目は愛人の
   セルビア・カエピオニス、では3人目は誰か? 超有名人\\
\vspace{1zw}A  クレオパトラ
   シーザーとの間にカエサリオンという子をもうけたという説もある。
}
\item{
「絨毯の中からカエサルの前へ現れるクレオパトラ」という名画
   ジャン=レオン・ジェローム画、1886 がある。絨毯の持つ意味は?

A 32 当時贈り物は絨毯に包んで贈った習慣があった。つまりクレオパトラ
   は宝物の替わりに自分の身を差し出した。そしてまんまと成功した。
}
\end{enumerate}
\end{document}
