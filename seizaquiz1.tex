\documentclass[fleqn]{article}
\usepackage{graphicx}
\usepackage{curves}
\usepackage{epic}
\usepackage{eepic}
\usepackage{amsmath}
\usepackage{amssymb}
%\usepackage{eepicemu}% bad!!
\usepackage[dvipdfmx,usenames]{color}
\usepackage{array}
\usepackage{colortbl}
\usepackage[usenames]{color}
\renewcommand{\arraystretch}{2.4}

\begin{document}
\title{星座 クイズ その1}
\maketitle
\hoffset = 0pt
\voffset = 0pt
\topmargin=0pt
\textheight=21cm
\headheight=0pt
\headsep=0pt
\newcommand{\CHo}{\makebox(0,0){$\bullet$}}
\newcommand{\CHc}{\makebox(0,0){$\circ$}}
\newcommand{\CHb}{\makebox(0,0){$\bigcirc$}}


\renewcommand{\descriptionlabel}[1]
{\hspace{\labelsep}\textsf{#1}}
\begin{description}
\item[Q 1]{大熊座の北斗七星のうち2等星はいくつあるか?}

\item[Q 2]{七夕伝説の牽牛星と織女星はこと座のヴェガと わし座
のアルタイルが対応します。牽牛星はどちらか?
}
\item[Q 3]{88星座のうち最も広い星座は?}

\item[Q 4]{領域がつながっていない星座はどれか?
またそれを分断した星座は?
}

\item[Q 5]{北斗七星の柄のカーブをそのまま南に延ばして得られる
「春の大曲線」の主役の星2つを示せ。}

\item[Q 6]{秋の四辺形の3隅はペガサス座です。左上隅は何座の星か?}


\item[Q 7]{21ある1等星のtop three を示しなさい。}

\item[Q 8]{
オリオン座には四辺形とその中の
三ツ星があります。この7つの星のうち
で1等星を示しなさい}

\item[Q 9]{冬のダイヤモンドと呼ばれる6ケの星を
図で示しなさい。いずれも1等星です。

}
\item[Q10]{野尻抱影が「三ッ矢サイダ-」の商標から命名した
三ッ矢は何座にあるか?

}

\item[Q11]{紀元前3000年頃造られた、エジプトのギゼーのケオフス王の
大ピラミッドには、傾斜角27°、長さ116mのトンネルが北に向かって
いる。当時ある星を観測していたと思われる。その星は?


}


\item[Q12]{北斗七星の柄から2番目の星ミザールの傍に小さい星が
あり、アラビアでは兵士の眼の検査\footnote{
星座ガイドブック春夏編p49} にこの星の名は?
}

\item[Q13]{WikiPedeaでの「星座一覧」はラテン語名、略号 そして 日本語で
ソートできる。日本語でソートしたら一番始めにくる星座は?
}
\item[Q14]{元旦のTVで1月4日未明に 「しぶんぎ座で多くの流星が
みれる」と報じました。現在しぶんぎ座は存在しない。どこにあるのか?
}
\item[Q15]{動物の名前に大小をつけて対の星座があります。その
動物たちは?
}

\end{description}
\newpage


\setlength\unitlength{1truecm}






\newpage
{\Large 解答}

\begin{description}
%-----------------------------------------
\item[A 1]{ 北斗七星の2等星は6コある、。 fig 1\\

\begin{minipage}{5cm}
\begin{picture}(10,5)
\Thicklines
\begin{dashjoin}{,2}
\jput(9.9,4.3){\CHo α}
\jput(10.5,2.5){\CHo β}
\jput(7.9,1.0){\CHo γ}
\jput(6.0,2.3){\CHo δ}
\jput(4.5,2.2){\CHo ε}
\jput(2.8,2.3){\CHo ξ}
\jput(0.6,1.0){\CHo η}
\end{dashjoin}
\end{picture}
\end{minipage}

\begin{minipage}{3cm}
α Dobhe(2)\\
β Merach(2)\\
γ Phecda(2)\\
δ Megres(3)\\
ε Alioth(2)\\
ξ Mizar(2)\\
η Alcaido(2)
\end{minipage}
}
%-----------------------------------------
\item[A 2]{アルタイルが牽牛星でヴェガが織女星
}

\item[A 3]{うみへび座が 1302平方度で一番広い。全体\footnote{$ 2\pi r=360 , s=4\pi r^2$}の3\% にあたる
}



\item[A 4]{へび座が頭部と尾部に分断されている。分断」したのは「へびつかい座」。
}
\item[A 5]{
アークツルス(うしかい座)とスピカ(おとめ座)
}
\item[A 6]{アンドロメダ座のα
}

\item[A 7]{シリウス、カノープス、リゲル・ケンタルス

\newpage


\item[A 8]
{
Orionには 1等星は2ツある\\
\begin{minipage}{7cm}

\begin{picture}(10,8.5)
\put(0.7,7.7){\CHo α}
\put(4.5,1.6){\CHo β}
\put(3.5,7.3){\CHo γ}
\put(3.0,4.8){\CHo δ}
\put(2.6,4.3){\CHo ε}
\put(2.1,4.0){\CHo ξ}
\put(1.4,1.0){\CHo κ}
\end{picture}
\end{minipage}
\begin{minipage}{5cm}{
 α Betercuse(1)}\\
 β Rigel(1)\\
 γ Bellatrix(2)\\
 δ Mintaka(2)\\
 ε Alnilam(2)\\
 ξ Alnitak(2)\\
 κ Saiph(2)
  \end{minipage}



}

\item[A 9]{

冬のダイヤモンド\\

\begin{minipage}{3cm}

\begin{picture}(10,10)
\begin{dashjoin}{,2}
\jput(9.6 ,7.4){\CHo 1}
\jput(7.1,9.6){\CHo 2}
\jput(0.4,9.4){\CHo 3}
\jput(0.7,5.0){\CHo 4}
\jput(3.4,0.8){\CHo 5}
\jput(7.7,2.5){\CHo 6}
\jput(9.6,7.4){ }
\end{dashjoin}
\end{picture}
\end{minipage}
\begin{minipage}{4cm}
1 Aldebaran(Boo おうし)\\
2 Capella(Aur ぎょしゃ)\\
3 Pollux(Gem ふたご)\\
4 Procyon(CMi こいぬ)\\
5 Sirius(CMa おおいぬ)\\
6 Rigel(Ori オリオン)
\end{minipage}
}


}
\newpage

\item[A 10]{ みずがめ座 \\
\begin{minipage}{6cm}

\begin{picture}(10,2.5)
\begin{dashjoin}{,2}
\jput(1.4,1.3){}
\jput(2.1,0.8){\CHo γ}
\end{dashjoin}
\begin{dashjoin}{,2}
\jput(0.9 ,1.2){\CHo η}
\jput(1.4,1.3){\CHo ξ}
\jput(1.9,1.9){\CHo π}
\end{dashjoin}
\begin{dashjoin}{,2}
\jput(3.5,1.3){\CHo α}
\jput(4.9,0.6){\CHo o}
\end{dashjoin}



\end{picture}
\end{minipage}




}
\item[A 11]{ りゅう座のα ツバーン。}
\item[A 12]{アルゴル}
\item[A 13]{いっかくじゅう座。アンドロメダ座はカタカナなので「ひらがな」の後。
}
\item[A 14]{
四分儀座は1922年に88星座を決めた時に、選から外され、
しぶんぎ座が設定されていた領域はりゅう座の一部となりました。
}

%-----------------------------------------

\item[A 15]{熊、犬、獅子。
略称の後ろ2文字は major と minor を示すMa 、Miが付きます。
例外はしし。

{\renewcommand\arraystretch{1.2}
\begin{tabular}{lc}
おおくま&UMa\\
こぐま&UMi\\
おおいぬ&CMa\\
こいぬ&CMi\\
しし&Leo\\
こじし&LMi\\
\end{tabular}
}%renew

}





%---------------------------------
\end{description}





\end{document}

